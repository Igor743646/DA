\section{Описание}

Длинная арифметика - это набор арифметических операций, такие как: сложение, вычитание, умножение, деление и т.д., выполняющихся над числами, разрядность которых превышает длину машинного слова. Эти операции выполняются программно с использованием стандартных аппаратных средств работы с числами меньших разрядов.

Для реализации длинных чисел я создал класс BigInt, полем которого является вектор, каждый элемент которого является разрядом в 10000 системе счисления. Конструктор принимает строку, которая в последствии парсит символьное представление числа в вектор разрядов.

\begin{lstlisting}[language=c++]
namespace Calculater {
    class BigInt {

        static const int32_t BASE = 10000;
        static const int32_t RADIX = 4;

    private:
        std::vector<int32_t> _data;

    public:

        BigInt() = default;

        BigInt(const std::string& str) {
            for (size_t i = str.size(); i > 0; i -= BigInt::RADIX) {
                if (i < BigInt::RADIX) {
                    _data.push_back(atoi(str.substr(0, i).c_str()));
                    break;
                } else {
                    _data.push_back(atoi(str.substr(i - BigInt::RADIX, BigInt::RADIX).c_str()));
                }     
            }

            RemoveZeros();
        }
        
        BigInt(const int32_t num) {
            _data.push_back(num % BigInt::BASE);
            _data.push_back((num / BigInt::BASE) % BigInt::BASE);
            _data.push_back((num / BigInt::BASE) / BigInt::BASE);

            RemoveZeros();
        }

        void RemoveZeros() {
            while (_data.size() > 1 && _data.back() == 0) {
                _data.pop_back();
            }
        }
    }
}
\end{lstlisting}
\pagebreak
Сложение и вычитание реализуются на подобии алгоритмов выполнения этих операций в столбик и имеют сложность $O(max\{n, m\})$, где n и m - число разрядов чисел.

\begin{lstlisting}[language=c++]
BigInt operator+(const BigInt& a) const {
            BigInt res;
            int32_t carry = 0;
            size_t n = std::max(a._data.size(), _data.size());
            for (size_t i = 0; i < n; ++i) {
                int32_t sum = carry;
                if (i < a._data.size()) {
                    sum += a._data[i];
                }
                if (i < _data.size()) {
                    sum += _data[i];
                }
                carry = sum / BigInt::BASE;
                res._data.push_back(sum % BigInt::BASE);
            }
            if (carry != 0) {
                res._data.push_back(1);
            }
            res.RemoveZeros();
            return res;
        }

BigInt operator-(const BigInt& a) const {
            BigInt res;
            int32_t p, carry = 0;
            size_t n = std::max(a._data.size(), _data.size());
            for (size_t i = 0; i < n; ++i) {
                p = _data[i] - carry;
                carry = 0;

                if (i >= a._data.size()) {
                    if (p < 0) {
                        carry = 1;
                        p += BigInt::BASE;
                    }
                    res._data.push_back(p);
                    continue;
                }

                if (p < a._data[i]) {
                    carry = 1;
                    p += BigInt::BASE;
                }

                res._data.push_back(p - a._data[i]);
            }

            res.RemoveZeros();
            return res;
        }
\end{lstlisting}
\pagebreak
Умножение тоже похоже на умножение в столбик, однако мы не будем складывать в конце все числа, полученные перемножением по разрядам, а будем сразу считать результат с учетом сдвигов при переполнении разрядов. Так же важно упомянуть, что для 10000 разрядности умножение двух разрядов не выйдет за пределы 32-битного типа int. Сложность наивного алгоритма умножения $O(n*m)$, что не очень хорошо, когда количество разрядов числа слишком велико, поэтому для более сложных случаев применяются алгоритмы Карацубы или Шёнхаге-Штрассена.

\begin{lstlisting}[language=c++]
BigInt operator*(const BigInt& a) const {
            BigInt res;
            size_t n = a._data.size() + _data.size();
            res._data.resize(n);

            int32_t k = 0;
            int32_t r = 0;
            for (size_t i = 0; i < _data.size(); i++) {
                for (size_t j = 0; j < a._data.size(); j++) {
                    k = _data[i] * a._data[j] + res._data[i + j];
                    r = k / BigInt::BASE;
                    res._data[i + j + 1] = res._data[i + j + 1] + r;
                    res._data[i + j] = k % BigInt::BASE;
                }
            }

            res.RemoveZeros();
            return res;
        }
        
BigInt operator*(const int32_t& a) const {
            BigInt c(a);
            return c * (*this);
        }
\end{lstlisting}

Быстрое возведение в степень - алгоритм, учитывающий четность степени, позволяет возводить число со сложностью $O(log{n})$, где n - количество перемножений, которые надо совершить.

\begin{lstlisting}[language=c++]
bool Odd(const BigInt& a) const {
            return a._data[0] & 1;
        }

BigInt Pow(BigInt a, BigInt b) const {
            BigInt res(1);
            while (b > BigInt(0)) {
                if (Odd(b)) {
                    res = res * a;
                }
                a = a * a;
                b = b / BigInt(2);
            }
            return res; 
        }
\end{lstlisting}
\pagebreak
Деление является, пожалуй, самым сложным алгоритмом. Я реализовал деление столбиком. Его примерная сложность $O((n-m)*log{BASE}*(m))$, так как по реализации цикл проходит $n-m$ раз, внутри цикла бинарным поиском подбирается нужное значение разряда в пределах от 1 до 10000, и при каждом подборе мы производим умножение $m$ на разряд, т.е. сложность умножения в данном случае будет $O(m)$.

\begin{lstlisting}[language=c++]
BigInt operator/(const BigInt& a) const {
            BigInt res;
            
            if (*this == a) return BigInt(1);
            if (*this < a) return BigInt(0);

            if (_data.size() == a._data.size()) {
                return BigInt(FindFactor(*this, a));
            } else {
                BigInt cur;
                cur._data = {_data.begin() + (_data.size() - a._data.size()), _data.end()};
                int32_t inc = _data.size() - a._data.size();

                if (cur < a) {
                    cur._data.insert(cur._data.begin(), _data[inc - 1]);
                    inc--;
                }
                do {
                    int32_t num = FindFactor(cur, a);
                    res._data.insert(res._data.begin(), num);
                    cur = cur - (a * num);
                    if (inc == 0) {
                        break;
                    } else {
                        inc--;
                        cur._data.insert(cur._data.begin(), _data[inc]);
                    }
                    cur.RemoveZeros();
                } while(true);
            }
            
            return res;
        }
\end{lstlisting}

Операции сравнения реализуются очень просто, сначала сравниваются длины векторов, и в случае если они равны, то поразрядно. Поэтому их код я приводить не буду.

\pagebreak
