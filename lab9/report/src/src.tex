\section{Описание}

Неориентированный граф - это совокупность двух множеств: множеств элементов-вершин графа и их парные связи - рёбра. Компонентой связности графа называется максимальный связный подграф данного графа, т.е. такой подграф, в котором из любой вершины существует путь к любой другой её вершине. Для нахождения компонент связности можно использовать простой поиск в глубину или ширину. 
Для каждой вершины, в которой мы еще не побывали ни разу, запускаем обход в ширину, записываем, что вершины обхода принадлежат одной компоненте, заносим эти вершины в контейнер посещенных и продолжаем так, пока все вершины не будут обойдены. Таким образом мы поучим список компонент связности, представленных тоже списком, уже вершин. 

\newpage

\section{Исходный код}

Класс Графа содержит хэш-таблицу, где ключом является номер вершины, а значением множество вершин, в которые есть путь из данной. Метод AddEdge добавляет ребро неориентированного графа. Метод Components возвращает список компонент связности.

\begin{lstlisting}[language=c++]
class Graph {

	using vertex_type = size_t;

	public:
		unordered_map<vertex_type, set<vertex_type>> edges;

	Graph() = default;
	Graph(const size_t n) {
		for (size_t i = 1; i <= n; i++) 
			edges[i] = set<vertex_type>{i};
	}

	void AddEdge(const vertex_type& from, const vertex_type& to) {
		edges[from].insert(to);
		edges[to].insert(from);
	}

	vector<vector<vertex_type>> Components() {
		vector<vector<vertex_type>> result;
		map<vertex_type, vertex_type> visited;
		vertex_type current;

		for (const auto& vt : edges) {
			if (visited.find(vt.first) == visited.end()) {

				current = vt.first;
				vector<vertex_type> p;
				queue<vertex_type> queue;

				queue.push(current);
		
				while (!queue.empty()) {
					current = queue.front();
					queue.pop();

					if (visited.find(current) == visited.end()) {
						visited.insert({current, current});
						p.push_back(current);

						for (const vertex_type& ver : edges[current]) {
							queue.push(ver);
						}
					}
				}

				sort(p.begin(), p.end());
				result.push_back(p);

			}
		}

		sort(result.begin(), result.end(), 
		[](const vector<vertex_type>& v1, const vector<vertex_type> v2) {
			return v1 < v2;
		});

		return result;
	}
};
\end{lstlisting}

В main считывается количество вершин и ребер, затем создается граф из n-1 вершины и добавляются ребра. Затем ищем компоненты связности и печатаем на экран.

\begin{lstlisting}[language=c++]
int main() {

	size_t n = 0, m = 0;
	cin >> n >> m;

	Graph gr(n);

	for (size_t i = 0; i < m; i++) {
		size_t from, to;
		cin >> from >> to;
		gr.AddEdge(from, to);
	}

	print(gr.Components());

	return 0;
}
\end{lstlisting}

\pagebreak

\section{Консоль}
\begin{alltt}
igor@igor-Aspire-A315-53G:~/Рабочий стол/c++/DA/lab9$ ./main.out
5 4
1 2
2 3
1 3
4 5
1 2 3 
4 5
igor@igor-Aspire-A315-53G:~/Рабочий стол/c++/DA/lab9$ ./main.out
5 4
1 1
1 1
1 1
1 1
1 
2 
3 
4 
5 
igor@igor-Aspire-A315-53G:~/Рабочий стол/c++/DA/lab9$
\end{alltt}
\pagebreak
