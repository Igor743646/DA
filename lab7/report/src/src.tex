\section{Описание}

Динамическое программирование - это способ решения сложных задач путем разбиения на более простые.

В моем задании наивное решение предполагает рекурсивный вызов по рекуррентной формуле: 

\begin{equation*}
  f(n) = 
  \begin{cases}
      0, &\text{$n = 1$} \\
      min\{f(n-1) + n; f(n/2) + n; f(n/3) + n\}, &\text{$n\ \vdots\ 2$, $n\ \vdots\ 3$} \\
      min\{f(n-1) + n; f(n/2) + n\}, &\text{$n\ \vdots\ 2$} \\
      min\{f(n-1) + n; f(n/3) + n\}, &\text{$n\ \vdots\ 3$} \\
      f(n-1) + n, &\text{}
  \end{cases}
\end{equation*}

Вычислим примерно сложность такого решения: 

Для каждого n можно построить дерево вызовов. Пусть функция g(n) возвращает количество вершин этого дерева. 

$g(1) = 1$; \\
$g(2) = g(1) + g(1) + 1 = 3$; \\
$g(3) = g(2) + g(1) + 1 = 5$; \\
\vdots \\
$g(6) = g(5) + g(3) + g(2) + 1 = 18$; \\
\vdots \\
$g(n) \leq g(n - 1) + g(n / 2) + g(n / 3) + 1;$ \\


\begin{center}
  $g(n / 2) < g(n - 1)$ \\
  $g(n / 3) + 1 < g(n - 1)$
\end{center}
$=> g(n) < 3g(n - 1) \\
=> g(n) < 3^{2}g(n-2) \\
=> g(n) < 3^{3}g(n-2) \\
=> g(n) < 3^{n-1}g(1) \\
=> g(n) < 3^{n-1}$

Отсюда следует, что верхняя оценка сложности наивного алгоритма $O(3^n)$. При нижней оценке мы предполагаем, что рекурсия вызывается только для $g(n-1)$, поэтому она будет равняться $O(n)$.

Эту задачу можно решить эффективнее за $O(n)$ в любом случае. Для этого мы избавимся от одних и тех же подсчетов f(n). Введем массив длины $n$, где для каждого числа будем хранить минимальную стоимость. 
Вычислять эти значения мы будем восходящим путем лишь единожды, за счет чего будет обеспечена линейная сложность как по времени, так и по памяти. В итоге результат будет храниться в последней ячейке массива.

\newpage

\section{Исходный код}

\begin{lstlisting}[language=c++]
unsigned int n;
std::cin >> n;

unsigned int* table = new unsigned int[n]();

for (unsigned int ind = 1; ind < n; ind++) {
	unsigned int number = ind + 1;

	if (number % 2 == 0 and number % 3 == 0) {
		if (table[ind - 1] < table[ind / 2]) {
			if (table[ind - 1] < table[ind / 3]) {
				table[ind] = table[ind - 1] + number;
			} else {
				table[ind] = table[ind / 3] + number;
			}
		} else {
			if (table[ind / 2] < table[ind / 3]) {
				table[ind] = table[ind / 2] + number;
			} else {
				table[ind] = table[ind / 3] + number;
			}
		}
	} else if (number % 3 == 0) {
		if (table[ind - 1] < table[ind / 3]) {
			table[ind] = table[ind - 1] + number;
		} else {
			table[ind] = table[ind / 3] + number;
		}
	} else if (number % 2 == 0) {
		if (table[ind - 1] < table[ind / 2]) {
			table[ind] = table[ind - 1] + number;
		} else {
			table[ind] = table[ind / 2] + number;
		}
	} else {
		table[ind] = table[ind - 1] + number;
	}
}
\end{lstlisting}

Востановление результата происходит с конца. Мы знаем, что стоимость на последнем шаге ровно на $n$ больше, чем на предыдущем. Поэтому из конечной стоимости вычитается $n$ и уже новая стоимость ищется в табличке. Найдя её мы узнаем, какую операцию мы совершили и продолжаем так, пока не дойдем до n = 0.

\begin{lstlisting}[language=c++]
std::string res = "";

unsigned int last_n = n;
unsigned int new_n = n;
long int i = n - 1;
unsigned int delta = table[i] - (i + 1);

while (i >= 0) {
	if (delta == table[i]) {
		last_n = i + 1;

		if (last_n * 3 == new_n) {
			res += "/3 ";
		} else if (last_n * 2 == new_n) {
			res += "/2 ";
		} else if (last_n + 1 == new_n) {
			res += "-1 ";
		} else {
			i--;
			continue;
		}

		new_n = last_n;
		delta -= last_n;
	}

	i--;
}

std::cout << table[n-1] << std::endl;
std::cout << res << std::endl;
\end{lstlisting}

\pagebreak

\section{Консоль}
\begin{alltt}
igor@igor-Aspire-A315-53G:~/Рабочий стол/c++/DA/lab7$ ./main.out
576
894
/3 /3 /2 /2 /2 /2 /2 /2 
igor@igor-Aspire-A315-53G:~/Рабочий стол/c++/DA/lab7$ ./main.out
123456789
208825723
/3 /3 -1 /2 /2 -1 /3 -1 /3 /3 -1 /2 /2 -1 /3 /3 /3 /3 /2 /2 /2 -1 /3 /2 /2 /2 /2 
igor@igor-Aspire-A315-53G:~/Рабочий стол/c++/DA/lab7$
\end{alltt}
\pagebreak
