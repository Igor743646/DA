\CWHeader{Лабораторная работа \textnumero 7}

\CWProblem{
При помощи метода динамического программирования разработать алгоритм решения задачи, определяемой своим вариантом; оценить время выполнения алгоритма и объем затрачиваемой оперативной памяти. Перед выполнением задания необходимо обосновать применимость метода динамического программирования.

Разработать программу на языке C или C++, реализующую построенный алгоритм. Формат входных и выходных данных описан в варианте задания:

Имеется натуральное число n. За один ход с ним можно произвести следующие действия: вычесть единицу, разделить на два, разделить на три. При этом стоимость каждой операции — текущее значение n. Стоимость преобразования — суммарная стоимость всех операций в преобразовании. Вам необходимо с помощью последовательностей указанных операций преобразовать число n в единицу таким образом, чтобы стоимость преобразования была наименьшей. Делить можно только нацело.
}

\pagebreak
