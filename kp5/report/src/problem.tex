\CWHeader{Задача}

\CWProblem{
Реализовать систему для определения принадлежности точки одному из многоугольников на плоскости.

{\bf ./prog index \--\--input <input file> $\backslash$

 \ \ \ \ \ \ \ \ \ \ \ \ \ \ \ \ \ \--\--output <index file>
}

\begin{table}[h]
  \begin{tabular}{|l |l |}
    \hline
    Ключ & Значение \\
    \hline
    \--\--input & входной файл с многоугольниками \\
    \hline
    \--\--output & выходной файл с индексом \\
    \hline
  \end{tabular}
\end{table}

{\bf ./prog search \--\--index <index file> $\backslash$

\ \ \ \ \ \ \ \ \ \ \ \ \ \ \ \ \ \ \--\--input <input file> $\backslash$
 
\ \ \ \ \ \ \ \ \ \ \ \ \ \ \ \ \ \ \--\--output <output file>
}

\begin{table}[h]
  \begin{tabular}{|l |l |}
    \hline
    Ключ & Значение \\
    \hline
    \--\--index & входной файл с индексом \\
    \hline
    \--\--input & входной файл с запросами \\
    \hline
    \--\--output & выходной файл с ответами на запросы \\
    \hline
  \end{tabular}
\end{table}

Формат входного файла:

{\bf <количество многоугольников> <количество вершин многоугольника [n]> <$x_1$> <$y_1$> <$x_2$> <$y_2$> ... <$x_n$> <$y_n$>
}

Формат файла запросов:

{\bf <$x_i$> <$y_i$>
}

Для каждого запроса вывести номер многоугольника, внутри которого содержится точка (многоугольники нумеруются с единицы), либо -1.

}

\pagebreak
