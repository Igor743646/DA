
\section{Выводы}
В ходе второй лабораторной работы я изучил инструменты gnuplot и perf. Gnuplot оказался не таким уж и сложным, а скорее даже приятным для построения графиков и анализа 
сложностей алгоритмов. Надеюсь, я и дальше не поленюсь и буду пользоваться данным инструментом. Perf оказался сложной утилитой. Для того, чтобы сделать тривиальные вещи
пришлось потратить какое-то время, однако эта сложность определяет мощь всего инструмента. Утилита показывает все узкие горлышки программы и анализирует вплоть до кода на
ассемблере. Благо из-за простоты поиска в интернете, можно изучить, за что именно отвечает кусок кода и примерно понять, как оптимизировать код, если на то будут причины.
Valgrind старый инструмент, которым я пользуюсь, а в данной лабораторной работе даже не оказалось ошибок утечек памяти, однако при написании второй лабораторной работы эта 
утилита сильно экономит время и помогает предотвратить ошибки, которые могут выявиться только спустя какое-то время. В моем опыте были ситуации, что программы, написанные
на ОС Linux Ubuntu не происходило никаких ошибок, однако при портировании кода на Windows 10 они случались. Оказалось, что Ubuntu снисходительнее относится к использованию
неинициализированной памяти и поэтому на ней сложнее было находить утечки, однако Valgrind позволяет мониторить данные ситуации и вовремя замечать ошибки. Так же в данной
лабораторной работе пришлось вспомнить bash, который тоже бывает полезен при разработке сложных программ и тестов для них. Результаты эффективности программы также оказались
неожиданностью. Я совсем не ожидал увидеть такую скорость вставки в AVL-дерево, однако это можно было предположить, так как на чекере моя программа на самом длительном 13 
тесте прошла за 40 секунд, в то время как максимально на выполнение дается 130 секунд. 
\pagebreak
