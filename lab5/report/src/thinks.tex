
\section{Выводы}
В ходе пятой лабораторной работы я узнал еще один из многочисленных способов поиска подстроки в строке с препроцессингом. Реализовал алгоритм Укконена, позволяющий построить суффиксное дерево за линейную сложность. Сначала было непонятно, с чего вообще начать, но обратив внимание на тот факт, что дерево можно представить графом, рассмотрев англоязычные статьи с псевдокодами и визуализатор алгоритма, я понял, что именно нужно делать. Обрадовало небольшое количество операций с ссылками, из-за чего программе потребовалось очень мало времени на дебаг утечек памяти. 
\pagebreak
