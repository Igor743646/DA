\section{Описание}

Суффиксное дерево - это особая структура данных, содержащая все суффиксы исходной строки, по которой это дерево было построено. Такое дерево позволяет искать подстроку в строке за
линейное время, пропорциональное длине заданной подстроки.

Свойства суффиксного дерева:
\begin {itemize}
  \item Количество листьев равняется количеству букв в исходной строке
  \item У каждой внутренней вершины есть хотя бы два ребенка.
  \item Каждое ребро помечено подстрокой из исходной строки
  \item Никакие два ребра, выходящие из одной вершины, не имеют пометки, начинающиеся с одинакого символа
  \item Дерево содержит все суффиксы исходной строки, причем все они заканчиваются в листе и больше нигде
  \item При построении дерева созданный на каком-либо шаге лист всегда останется листом вплоть до окончания построения дерева.
\end{itemize}

\pagebreak
\section{Исходный код}

Сначала, заметив, что в суффиксном дереве у вершины неизвестное число потомков, я решил, что данное дерево будет удобнее представлять в виде графа, поэтому вершины храняться в целостном массиве, где индекс в массиве является ее номером. Мы знаем, что число вершин в дереве не превосходит $2(n-1)$, поэтому мы сразу можем выделить память под максимальное количество вершин. Такое представление помогло меньше думать об утечках памяти и представлять все суффиксные и обычные ссылки как номера типа long. 

В дереве дополнительно описан класс итератора, с помощью которого оно будет строится. Он имеет пять полей: вершина, в которой сейчас находится; индекс символа и сам символ, который вставляется; счетчек дополнительных вставок; и длинна подстроки (суффикса), которую нужно вставить.

Сама структура хранит указатель на исходную строку, массив вершин, глобальную переменную end для построения дерева и количество вершин. 

\begin{lstlisting}[language=c++]
namespace Tree {

    class SufTree {

    public:

        class Node {
            public:
                long l, r, index;
                long suf_link = 0;
                long str_index;
                std::unordered_map<char, long> next_vertexes;

                Node(const long _l = -1, const long _r = -1, const long _index = -1, 
                const long _str_index = -1): l(_l), r(_r), index(_index), 
                str_index(_str_index) {}

                long Length(const long end) const {
                    return std::min(r, end) - l;
                }

        };

        class Iterator {
            public:
                Node* active_node;
                long active_edge_index;
                long active_length;
                unsigned long remainder;
                char active_edge_char;

                Iterator(Node* _active_node, long _active_edge_index,
                char _active_edge_char, long _active_length, long _remainder): 
                active_node(_active_node), active_edge_index(_active_edge_index), 
                active_length(_active_length), remainder(_remainder), 
                active_edge_char(_active_edge_char) {}

                void CreateEdge(const unsigned long node_count) {
                    active_node->next_vertexes[active_edge_char] = node_count;
                }

                void FirstRule(const std::string* text) {
                    active_edge_index++;
                    if (active_edge_index < text->size())
                        active_edge_char = (*text)[active_edge_index];
                    active_length--;
                }

                void ThirdRule(const Node* vertexes) {
                    active_node = (Node*) vertexes + active_node->suf_link;
                }

        };
        
        long end = 0;
        const std::string* text;
        Node* vertexes;
        unsigned long node_count = 1;
};
\end{lstlisting}

В main'е сначала считывается текст, на его основе строится суффиксное дерево. Далее в цикле while считываются паттерны, для которых надо найти вхождения. В метод суффиксного дерева Find я подаю указатель на вектор, чтобы его заполнили номерами позиций, с которых паттерн начинается. После этого массив сортируется, выводится ответ и очищаетя для следующего запроса. 

\begin{lstlisting}[language=c++]
int main() {

    std::ios::sync_with_stdio(false);
    std::cin.tie(0);
    std::cout.tie(0);

    std::string text;
    std::cin >> text;
    text += '$';
    std::vector<long> occurrences;
    
    Tree::SufTree st(&text);

    /*--------------------------------------------*/

    std::string pattern;
    unsigned long pattern_index = 1;

    while (std::cin >> pattern) {
        st.Find(&pattern, &occurrences);

        if (occurrences.size() == 0) {
            pattern_index++;
            continue;
        }

        std::stable_sort(occurrences.begin(), occurrences.end());

        printf("%ld: ", pattern_index);
        for (size_t j = 0; j < occurrences.size() - 1; ++j) {
           printf("%ld, ", occurrences[j]);
        }
        
        printf("%ld\n", occurrences.back());

        pattern_index++;
        occurrences.clear();
    }

    return 0;
}
\end{lstlisting}

\begin{longtable}{|p{7.5cm}|p{7.5cm}|}
\hline
\rowcolor{lightgray}
\multicolumn{2}{|c|} {main.cpp}\\
\hline
Tree::SufTree::Node::Node (...) & Конструктор класса Node\\
\hline
long Tree::SufTree::Node::Length (const long end) const  & Метод узла, возращающий длину ребра\\
\hline
Tree::SufTree::Iterator::Iterator (...)  & Конструктор класса итератор\\
\hline
void Tree::SufTree::Iterator::CreateEdge (const unsigned long node\_count) & Создает переход по таблице в вершину с номером node\_count\\
\hline
void Tree::SufTree::Iterator::FirstRule (const std::string* text)  & Первое правило для случая, когда активная вершина корневая\\
\hline
void Tree::SufTree::Iterator::ThirdRule (const Node* vertexes)  & Третье правило для случая, когда активная вершина не корневая\\
\hline
Tree::SufTree::SufTree(const std::string* t)  & Конструктор суффиксного дерева\\
\hline
void Tree::SufTree::FindIndexes(const Node* cur, std::vector<long>* r) const  & Заполняет вектор r всеми индексами листьев поддерева с корнем cur\\
\hline
bool Tree::SufTree::Find(const std::string* pattern, std::vector<long>* r) & Ищет все вхождения паттерна и заносит их в вектор r\\
\hline
Tree::SufTree::\texttildelow SufTree() & Деструктор\\
\hline
\end{longtable}

\pagebreak
\section{Консоль}
\begin{alltt}
igor@igor-Aspire-A315-53G:~/Рабочий стол/c++/DA/lab5$ ./main2.out
abcdabc
abcd
1: 1
bcd
2: 2
bc
3: 2, 6
igor@igor-Aspire-A315-53G:~/Рабочий стол/c++/DA/lab5$
\end{alltt}
\pagebreak
